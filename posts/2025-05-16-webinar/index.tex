% Options for packages loaded elsewhere
\PassOptionsToPackage{unicode}{hyperref}
\PassOptionsToPackage{hyphens}{url}
\PassOptionsToPackage{dvipsnames,svgnames,x11names}{xcolor}
%
\documentclass[
  letterpaper,
  DIV=11]{scrartcl}

\usepackage{amsmath,amssymb}
\usepackage{iftex}
\ifPDFTeX
  \usepackage[T1]{fontenc}
  \usepackage[utf8]{inputenc}
  \usepackage{textcomp} % provide euro and other symbols
\else % if luatex or xetex
  \usepackage{unicode-math}
  \defaultfontfeatures{Scale=MatchLowercase}
  \defaultfontfeatures[\rmfamily]{Ligatures=TeX,Scale=1}
\fi
\usepackage{lmodern}
\ifPDFTeX\else  
    % xetex/luatex font selection
\fi
% Use upquote if available, for straight quotes in verbatim environments
\IfFileExists{upquote.sty}{\usepackage{upquote}}{}
\IfFileExists{microtype.sty}{% use microtype if available
  \usepackage[]{microtype}
  \UseMicrotypeSet[protrusion]{basicmath} % disable protrusion for tt fonts
}{}
\makeatletter
\@ifundefined{KOMAClassName}{% if non-KOMA class
  \IfFileExists{parskip.sty}{%
    \usepackage{parskip}
  }{% else
    \setlength{\parindent}{0pt}
    \setlength{\parskip}{6pt plus 2pt minus 1pt}}
}{% if KOMA class
  \KOMAoptions{parskip=half}}
\makeatother
\usepackage{xcolor}
\setlength{\emergencystretch}{3em} % prevent overfull lines
\setcounter{secnumdepth}{-\maxdimen} % remove section numbering
% Make \paragraph and \subparagraph free-standing
\makeatletter
\ifx\paragraph\undefined\else
  \let\oldparagraph\paragraph
  \renewcommand{\paragraph}{
    \@ifstar
      \xxxParagraphStar
      \xxxParagraphNoStar
  }
  \newcommand{\xxxParagraphStar}[1]{\oldparagraph*{#1}\mbox{}}
  \newcommand{\xxxParagraphNoStar}[1]{\oldparagraph{#1}\mbox{}}
\fi
\ifx\subparagraph\undefined\else
  \let\oldsubparagraph\subparagraph
  \renewcommand{\subparagraph}{
    \@ifstar
      \xxxSubParagraphStar
      \xxxSubParagraphNoStar
  }
  \newcommand{\xxxSubParagraphStar}[1]{\oldsubparagraph*{#1}\mbox{}}
  \newcommand{\xxxSubParagraphNoStar}[1]{\oldsubparagraph{#1}\mbox{}}
\fi
\makeatother


\providecommand{\tightlist}{%
  \setlength{\itemsep}{0pt}\setlength{\parskip}{0pt}}\usepackage{longtable,booktabs,array}
\usepackage{calc} % for calculating minipage widths
% Correct order of tables after \paragraph or \subparagraph
\usepackage{etoolbox}
\makeatletter
\patchcmd\longtable{\par}{\if@noskipsec\mbox{}\fi\par}{}{}
\makeatother
% Allow footnotes in longtable head/foot
\IfFileExists{footnotehyper.sty}{\usepackage{footnotehyper}}{\usepackage{footnote}}
\makesavenoteenv{longtable}
\usepackage{graphicx}
\makeatletter
\def\maxwidth{\ifdim\Gin@nat@width>\linewidth\linewidth\else\Gin@nat@width\fi}
\def\maxheight{\ifdim\Gin@nat@height>\textheight\textheight\else\Gin@nat@height\fi}
\makeatother
% Scale images if necessary, so that they will not overflow the page
% margins by default, and it is still possible to overwrite the defaults
% using explicit options in \includegraphics[width, height, ...]{}
\setkeys{Gin}{width=\maxwidth,height=\maxheight,keepaspectratio}
% Set default figure placement to htbp
\makeatletter
\def\fps@figure{htbp}
\makeatother

\KOMAoption{captions}{tableheading}
\makeatletter
\@ifpackageloaded{caption}{}{\usepackage{caption}}
\AtBeginDocument{%
\ifdefined\contentsname
  \renewcommand*\contentsname{Inhaltsverzeichnis}
\else
  \newcommand\contentsname{Inhaltsverzeichnis}
\fi
\ifdefined\listfigurename
  \renewcommand*\listfigurename{Abbildungsverzeichnis}
\else
  \newcommand\listfigurename{Abbildungsverzeichnis}
\fi
\ifdefined\listtablename
  \renewcommand*\listtablename{Tabellenverzeichnis}
\else
  \newcommand\listtablename{Tabellenverzeichnis}
\fi
\ifdefined\figurename
  \renewcommand*\figurename{Abbildung}
\else
  \newcommand\figurename{Abbildung}
\fi
\ifdefined\tablename
  \renewcommand*\tablename{Tabelle}
\else
  \newcommand\tablename{Tabelle}
\fi
}
\@ifpackageloaded{float}{}{\usepackage{float}}
\floatstyle{ruled}
\@ifundefined{c@chapter}{\newfloat{codelisting}{h}{lop}}{\newfloat{codelisting}{h}{lop}[chapter]}
\floatname{codelisting}{Listing}
\newcommand*\listoflistings{\listof{codelisting}{Listingverzeichnis}}
\makeatother
\makeatletter
\makeatother
\makeatletter
\@ifpackageloaded{caption}{}{\usepackage{caption}}
\@ifpackageloaded{subcaption}{}{\usepackage{subcaption}}
\makeatother

\ifLuaTeX
\usepackage[bidi=basic]{babel}
\else
\usepackage[bidi=default]{babel}
\fi
\babelprovide[main,import]{ngerman}
% get rid of language-specific shorthands (see #6817):
\let\LanguageShortHands\languageshorthands
\def\languageshorthands#1{}
\ifLuaTeX
  \usepackage{selnolig}  % disable illegal ligatures
\fi
\usepackage{bookmark}

\IfFileExists{xurl.sty}{\usepackage{xurl}}{} % add URL line breaks if available
\urlstyle{same} % disable monospaced font for URLs
\hypersetup{
  pdftitle={Infra Wiss Blogs auf dem BID Kongress 2025},
  pdfauthor={Catharina Ochsner; Heinz Pampel},
  pdflang={de},
  colorlinks=true,
  linkcolor={blue},
  filecolor={Maroon},
  citecolor={Blue},
  urlcolor={Blue},
  pdfcreator={LaTeX via pandoc}}


\title{Infra Wiss Blogs auf dem BID Kongress 2025}
\author{Catharina Ochsner \and Heinz Pampel}
\date{2025-03-10}

\begin{document}
\maketitle


Wir freuen uns zu verkünden, dass das Projekt Infra Wiss Blogs auf dem
diesjährigen BID Kongress mit einem Vortrag sowie einem Hands-on Lab
vertreten sein wird.

Der Vortrag mit dem Titel
``\href{https://bid2025.abstractserver.com/program/\#/details/presentations/307}{Kartierung
deutscher Wissenschaftsblogs - Ergebnisse und Impulse}'' wird am 26.
Juni von 11 bis 11:30 in Halle 4.1/Raum II stattfinden.~

Catharina Ochsner, Heinz Pampel sowie Jonas Höfting vom
\href{https://www.ibi.hu-berlin.de/de/forschung/infomanagement}{Lehrstuhl
Information Management} des
\href{https://www.ibi.hu-berlin.de/de}{Instituts für Bibliotheks- und
Informationswissenschaft} der
\href{https://www.hu-berlin.de/de}{Humboldt-Universität zu Berlin} (HU
Berlin), werden die Ergebnisse einer Datenerhebung von über 800
deutschen Wissenschaftsblogs vor, die auf infrastrukturelle Aspekte
untersucht wurden, vorstellen. Darüber hinaus werden Impulse für
wissenschaftliche Bibliotheken gegeben, um einen Beitrag zur Sicherung
der dauerhaften Zugänglichkeit von Blogs zu leisten.

Das Hands-on Lab mit dem Titel
``\href{https://bid2025.abstractserver.com/program/\#/details/sessions/84}{Bibliotheken
als betreibende und bewahrende Institutionen für wissenschaftliche
Blogs}'' wird ebenfalls am 26. Juni von 14 bis 16 Uhr im Salon Oslo
stattfinden.~

Im Hands-on-Lab diskutieren die Teilnehmer:innen unter der Organisation
und Moderation von Expert:innen der wissenschaftlichen Bloglandschaft
die Rolle der Bibliotheken bei der Erschließung, Verbreitung und
Sicherung wissenschaftlicher Blogs. In Gesprächsrunden sollen Aspekte
von Technik, Organisation und Rahmenbedingungen der Sicherung von Blogs
diskutiert werden. Das Hands-on Lab wird geleitet von
\href{https://www.ibi.hu-berlin.de/de/forschung/infomanagement/teaminfomanagement}{Catharina
Ochsner} (HU Berlin) sowie unterstützt von
\href{https://heinzpampel.de/}{Heinz Pampel} (HU Berlin),
\href{https://verfassungsblog.de/author/elena-di-rosa/}{Elena Di Rosa}
(\href{https://verfassungsblog.de/}{Verfassungsblog}),
\href{https://front-matter.io/team}{Martin Fenner}
(\href{https://front-matter.io/}{Front Matter}), Sven Ködel
({[}Deutsches Historisches Institut Paris{]}
(https://www.dhi-paris.fr/home.html)),
\href{https://www.gesis.org/institut/ueber-uns/mitarbeitendenverzeichnis/person/Charmaine.Voigt}{Charmaine
Voigt} (\href{https://www.gesis.org/home}{GESIS: Leibniz Institut für
Sozialwissenschaften}) und
\href{https://www.dnb.de/SharedDocs/Kontaktdaten/DE/wolderingBritta.html}{Britta
Woldering} von der
\href{https://www.dnb.de/DE/Home/home_node.html}{Deutschen
Nationalbibliothek}. Die Anmeldung für den Workshop findet per
\href{mailto:catharina.ochsner@hu-berlin.de}{E-Mail} an Catharina
Ochsner statt.

Wir würden uns freuen, Sie auf dem BID Kongress zu sehen und mit Ihnen
über das Thema der langfristigen Verfügbarkeit von Wissenschaftsblogs zu
diskutieren. Bei Fragen zum BID Kongress oder den Veranstaltungen,
können Sie sich sehr gerne bei
\href{mailto:catharina.ochsner@hu-berlin.de}{Catharina Ochsner} melden.

Weitere Informationen über die Forschungsgruppe finden Sie auf unserer
\href{http://hu.berlin/infomgnt}{offiziellen Website}.

This text -- excluding quotes and otherwise labelled parts -- is
licensed under the
\href{https://creativecommons.org/licenses/by/4.0/deed.de}{CC BY 4.0
DEED}.




\end{document}
